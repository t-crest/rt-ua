%\documentclass[a4paper,twocolumn]{article}
%\documentclass[10pt, conference, compsocconf]{IEEEtran}
\documentclass[a4paper, conference]{IEEEtran}


\usepackage{pslatex} % -- times instead of computer modern, especially for the plain article class
\usepackage[colorlinks=false,bookmarks=false]{hyperref}
\usepackage{booktabs}
\usepackage{graphicx}
\usepackage{xcolor}
\usepackage{multirow}
\usepackage{cite}
%\usepackage{flushend} % even out the last page, but use only at the end when there is a bibliography

\newcommand{\code}[1]{{\small{\texttt{#1}}}}

% fatter TT font
\renewcommand*\ttdefault{txtt}
% another TT, suggested by Alex
% \usepackage{inconsolata}
% \usepackage[T1]{fontenc} % needed as well?

\usepackage{listings}

\newcommand{\todo}[1]{{\emph{TODO: #1}}}
\newcommand{\martin}[1]{{\color{blue} Martin: #1}}
\newcommand{\abcdef}[1]{{\color{red} Author2: #1}}

% uncomment following for final submission
%\renewcommand{\todo}[1]{}
%\renewcommand{\martin}[1]{}
%\renewcommand{\author2}[1]{}


%%Uncomment the following when you want to add copyright notice and not use any space	 (IEEE only)
%\usepackage[absolute]{textpos}
%% Set unit to be pagewidth and height, and increase inner margin of box
%\setlength{\TPHorizModule}{\paperwidth}\setlength{\TPVertModule}{\paperheight}
%\TPMargin{5pt}
%% Define \copyrightstatement command for easier use
%\newcommand{\copyrightstatement}{
%	\begin{textblock}{0.85}(0.072,0.93)    % Tweak here: {box width}(leftposition, rightposition)
%		\noindent
%		\normalsize
%		???-?-?-???-?/??/\$31.00~\copyright20?? IEEE % Put here your copyright
%	\end{textblock}
%}

\begin{document}

%%Uncomment the following when you want to add copyright notice and not use any space	 (IEEE only)
%\copyrightstatement

\title{Real-Time Automation with Universal Access (title ok?)}

\author{Proposal: Andreas Kirchberger, Thomas Fr�hwirth, Patrick Heinrich Denzler, Wolfgang Kastner, and  Martin Schoeberl}

% Most conferences have their own commands for author headings.

%\author{\IEEEauthorblockN{Edgar Lakis, Martin Schoeberl}\\
%\IEEEauthorblockA{Department of Applied Mathematics and Computer Science\\
%Technical University of Denmark\\
%Email: \texttt{edgar.lakis@gmail.com}, \texttt{masca@imm.dtu.dk}}
%}


\maketitle \thispagestyle{empty}

\begin{abstract}
A good paper has a short, concise abstract. The abstract
states the field of the research, the purpose, and the findings (results).
\end{abstract}

From IEEE: your abstract should:

* Provide a concise summary of the research conducted, the conclusions reached, and the potential implications of those conclusions

* Be self-contained, without abbreviations, footnotes, references, or mathematical equations

* Include 3-5 keywords or phrases that describe the research to help readers find your article

* Consist of a single paragraph of 250 words or less

* Communicate clearly and concisely, with correct grammar and unambiguous terminology

\begin{IEEEkeywords}
real-time systems, time-predictable computer architecture.
\end{IEEEkeywords}

\section{Notes}

The paper is about porting an OPC UA stack to Patmos~\cite{patmos:rts2018}
and provide time-predictable access via PubSub (? don't remember the exact term).
The whole stack shall be analyzable for the worst-case execution time (WCET).
We will use platin for WCET analysis~\cite{compiler:platin:kps15}.

\begin{itemize}
\item We will port open62541: https://open62541.org/ to Patmos
\item We may use tpIP~\cite{tpip:isorc2018}, but usability needs to be checked
\item Otherwise we could start with \emph{lwIP} (lightweight IP) or \emph{uIP} (micro IP)~\cite{uIP}.
\item For end-to-end predictability we need to use either TTEthernet or TNS with time-triggered configuration.
\end{itemize}

\subsection{Material and Links}

\begin{itemize}
\item Mahnke, Wolfgang, Stefan-Helmut Leitner, and Matthias Damm. OPC unified architecture. Springer Science \& Business Media, 2009.
\item \url{https://mycourses.aalto.fi/pluginfile.php/719938/mod_folder/content/0/OPC%20UA%20Mahnke%20Leitner%20Damm%202009.pdf?forcedownload=1}
\item Spezifikationen kann man gegen Registrierung hier herunterladen: \url{https://opcfoundation.org/developer-tools/specifications-unified-architecture}. Thomas recommends Part 1 (Overview and Concepts) ans Part 14 PubSub.
\item Thomas: Was die Software betrifft, ist die von Unified Automation ws. die fortschrittlichste: \url{https://www.unified-automation.com/}. Man kann die verschiedenen Softwarekomponenten jeweils gratis in einer Testversion ausprobieren. Normalerweise ben�tigt man zum Modellieren des Informationsmodells: UaModeler.
F�r die Entwicklung eines eigenen Servers: ein SDK in seiner bevorzugten Sprache.
Alternativ auch den Demo Server, den ich euch gestern gezeigt habe (unter Downloads -> OPC UA Servers).
\end{itemize}


\section{Introduction}
\label{sec:intro}

\todo{A brief introduction what the paper is about. It shall include briefly the
main contributions and findings. The contributions can be bullet listed.}

This paper... \todo{purpose statement, latest in 4th paragraph}

\todo{Test the bib with a reference that gives background on time-predictable
computer architecture~\cite{tpca:jes}.}

A paper is cited \cite{paper:example}.

The contributions of this paper are: (1) ... (2) ...

This paper is organized in N sections: The following section presents related work.
Section~\ref{sec:background} provides background on ...
Section X and Y 
Section~\ref{sec:eval} evaluates...
Section~\ref{sec:conclusion} concludes.

\section{Related Work}
\label{sec:related}

\todo{Show that you know the field. All related work shall be put
into context or contrast to our current work.}

\section{System Model}
\label{sec:sysmod}

\todo{Some RTS conferences like a system model description where we talk
on a set of periodic tasks...}

\section{Background}
\label{sec:background}

\todo{Some papers have a background section.}


\section{Main Section}
\label{sec:abc}

\todo{Here comes the main content. Can be split into two sections of needed.
E.g., a Design/Architecture and an Implementation section.}

\section{Evaluation}
\label{sec:eval}

\todo{Computer engineering is a constructive science. We build stuff and we measure.
Therefore, there shall always be an evaluation section.}

\subsection{Source Access}

\martin{I love doing papers with available source under an
open-source license. It gives credit and good karma.}

\section{Conclusion}
\label{sec:conclusion}

\todo{Rephrase what this paper is about and list the main contributions and results.}

\subsection*{Acknowledgement}

This project has received funding from the European Union's Horizon 2020
research and innovation programme under the Marie Sklodowska-Curie
grant agreement No. 764785.


\bibliographystyle{plain}
% Please do not add any references to msbib.bib.
% They get lost when I 'generate' is again (see Makefile)
\bibliography{rt-ua,msbib}

\end{document}
